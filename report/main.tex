\documentclass[a4paper]{article}
\usepackage[utf8]{inputenc}
\usepackage[T1]{fontenc}

\usepackage[backend=biber,sorting=none]{biblatex}
\addbibresource{references.bib}

\usepackage{amsmath}
\usepackage{appendix}
\usepackage{graphicx}
\usepackage{hyperref}
\usepackage{placeins}
\usepackage{subcaption}

\title{PAP328 project work: proportional counter}
\author{Mika Mäki}


% TODO add date when the report was handed in


\begin{document}

\maketitle

\section*{Abstract}
In this project we assembled a gas-based proportional counter for X-ray and $\gamma$-ray detection.
The detector was based on a 0,5 l drink can, which had an anode wire along its center.
The detector was filled with the P10 gas mixture, and high voltage ($>$ 2000 V) was applied between the anode and the can.
In these conditions when high-energy radation ionizes the gas, the resulting electrons drift towards the anode wire.
Close to the wire the field strength is high enough for avalanche formation, and the electrons are multiplied, resulting in an electrical pulse that can be detected with electronics.

The system was tested with Fe-55 and Am-241 sources, and a multichannel analyzer was used to collect the emission spectra, which were then analyzed with self-developed software.
This setup demonstrated sufficient resolution to distinguish the ASDF peaks.

The detector was also tested with a self-built pre-amplifier.
However, these results were inconclusive due to unexpected degradation of the detector.
These measurements are discussed in the appendix \ref{pre_amp}.


\section{Introduction}
What are radiation detectors?
Basics of proportional counters.

Context for the usages!


\section{Theory}
Formulas here and only here

\subsection{The proportional counter}
The detector.
Electric field, gas composition, quenching, multiplication, different regions.
Theoretical predictions on accuracy?

\subsection{Error analysis methods}
Error propagation formulas etc.


\section{Experimental set-up}
Self-built detector etc.
Answer the questions on page 5 of the instructions!


\subsection{Detector assembly}
Detector structure and construction steps

The top of the can was removed.
Holes.

\begin{figure}[ht!]
\centering
\includegraphics[width=\textwidth]{fig/IMG_20201117_121044.jpg}
\caption{The anode wire is soldered to the connector}
\end{figure}

\begin{figure}[ht!]
\centering
\includegraphics[width=\textwidth]{fig/IMG_20201123_103327.jpg}
\caption{A brass tube within a plastic screw serves as the other end of the anode wire}
\end{figure}

\begin{figure}[ht!]
\centering
\includegraphics[width=\textwidth]{fig/IMG_20201123_104201.jpg}
\caption{Setup for soldering the anode (TODO CHECK) wire}
\end{figure}


\FloatBarrier
\subsection{Calibration}
The electronics have to be calibrated with an external pulser

\begin{figure}[ht!]
\centering
\includegraphics[width=\textwidth]{fig/IMG_20201130_135000.jpg}
\caption{Setup for testing with an external pulser}
\end{figure}


\FloatBarrier
\subsection{Testing}

\begin{figure}[ht!]
\centering
\includegraphics[width=\textwidth]{fig/IMG_20201130_144418.jpg}
\caption{Setup for testing with an Fe-55 source and a rack-mounted amplifier}
\end{figure}


\FloatBarrier
\section{Results and discussion}


\subsection{Calibration}
Output vs. input signal
See the spreadsheet!

\subsection{High voltage sweep}
The effect of gain settings and detector voltages

\subsection{Spectral measurement}
Some nice MCA spectra here


\section{Conclusions}
Discuss results in context.
No need to repeat the results.
Refer to the introduction.
Was the aim fulfilled?
Were the results as expected? If not, why?
Is the mesurement precise enough to discuss ASDF?


\clearpage
\begin{appendices}

\section{Pre-amplifier}
\label{pre_amp}

Construction and detailed characterization here

\begin{figure}[ht!]
\centering
\includegraphics[width=\textwidth]{fig/IMG_20201005_104331.jpg}
\caption{Preliminary testing of the pre-amplifier components}
\end{figure}


\begin{figure}[ht!]
\centering
\subcaptionbox{}{
	\includegraphics[width=\textwidth]{fig/IMG_20201201_121635.jpg}
}
\subcaptionbox{}{
	\includegraphics[width=\textwidth]{fig/IMG_20201201_121625.jpg}
}
\caption{Power supply for the pre-amplifier}
\end{figure}

\begin{figure}[ht!]
\centering
\includegraphics[width=\textwidth]{fig/IMG_20201207_121010.jpg}
\caption{Pre-amplifier board}
\end{figure}

\begin{figure}[ht!]
\centering
\subcaptionbox{}{
	\includegraphics[width=\textwidth]{fig/IMG_20201201_121735.jpg}
}
\subcaptionbox{}{
	\includegraphics[width=\textwidth]{fig/IMG_20201201_121845.jpg}
}
\caption{Mounting of the pre-amplifier board}
\end{figure}

\end{appendices}


\clearpage
\section{References}
\printbibliography


\end{document}